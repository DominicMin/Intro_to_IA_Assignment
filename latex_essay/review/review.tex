\documentclass{xum_review}


\begin{document}



\tableofcontents
\newpage
\setcounter{page}{1}

\section{Introduction}
As Charles Darwin once said, "It is not the strongest of the species that
survive, nor the most intelligent, but the one most responsive to change." We
are all living in an ever-changing world. No matter how much we wish to hold on
to the familiar and comfortable environment around us, life always pushes us
onto new and exciting journeys. Sometimes, we may find it hard to get used to a
new environment and want to find helpful tools to help us adapt. 

One such tool is the sentiment-aware chatbot. When embedded in a campus context,
such a chatbot could not only provide timely information about the campus with
high accuracy, but also provide suitable emotional support for users. This
combination of sentimental analysis and campus context chatbot deserves deeper
exploration. Therefore, this review investigates the landscape of campus
chatbots and the sentiment analysis technology. To guide our systematic review,
we constructed the following research questions:
\begin{itemize}
    \item RQ1: What is the current state of chatbots in the current campus
    context?
    \item RQ2: What are the current technical approaches to sentimental
    analysis?
    \item RQ3: What are the impacts of sentimental analysis chatbot on users'
    experience?
\end{itemize}

To attain a focused and manageable scope for this literature review, we set the
inclusion criteria for studies. Studies were considered for inclusion if they
were peer-reviewed journal articles or full conference papers published between
2014 and 2024, written in any language (with priority given to English), and
directly focused on chatbot or conversational-agent technologies and/or
sentiment-analysis techniques. We excluded abstracts, posters, dissertations,
theses, commentaries, editorials, letters to the editor, and review articles, as
well as studies that only presented theoretical frameworks. Papers focusing on
applications outside the scope of conversational agents or sentiment
analysis, such as purely predictive analytics in finance, healthcare, or other
non-conversational domains--were also excluded.

\section{Landscape of Campus Chatbots}
In order to gain an in - depth understanding of the current state of chatbots in
universities, we have been conducting a comprehensive analysis of academic
literature. Particular emphasis has been placed on research findings that are
dedicated to chatbots in the university context. Research indicates that
chatbots have the potential to revolutionize campus technology, which frequently
lags behind the times. Additionally, they can confer multiple advantages to the
overall academic ecosystem, benefiting both students and faculty members alike
\citep{dibitonto2018chatbot}.

With the rapid development of artificial intelligence and natural language
processing technologies, educational chatbots have evolved into a diverse range
of specialized tools in academic settings. Modern educational chatbots can be
roughly classified based on their core functions and technical foundations.
These categories include teaching chatbots for academic assistance, emotional
support chatbots for promoting student well-being and Campus Service Chatbot for
facilitating campus interactions.

\subsection{Academic Assistance Chatbots}

The traditional teaching approach based on master classes or techniques, which
places students in a passive position, has been proven to be an inefficient
method in the learning process. The application of technology in universities
helps to facilitate students' interest, and enhance their engagement in their
own educational development. For this reason, the application of chatbot has
been brought into consideration. These techniques simulate the human thought
process by leveraging structures that encapsulate the knowledge and experience
of human experts \citep{villegas2020proposal}. One
example of academic assistance chatbot is EDUBOT.EDUBOT is an educational
chatbot developed by Sathyabama Institute of Science and Technology in India,
aiming to address the issue of the lack of teacher guidance during the online
learning of primary and secondary school students during the COVID-19 pandemic.
The system is built on the Google Dialogflow platform for semantic
understanding, combined with RNN models for language processing, and deployed on
the web via the Flask framework, supporting integration with mainstream social
platforms such as Telegram and Messenger. EDUBOT offers question-and-answer
services, a math calculator, demonstration videos, and the ability to save
history records. It is designed to be simple and user-friendly, targeting
students from grades 1 to 5, and provides 24-hour self-service answers \citep{sophia2021edubot}.

\subsection{Emotional Support Chatbots}

In Japan, due to COVID-19, the number of college students encountering
challenges has been on a rise. However, despite the availability of on-campus
counseling services, many students hesitate to use them. This reluctance is
primarily attributed to psychological barriers associated with seeking help
\citep{yasuda2021designing}. This highlights the essential role that emotional support
chatbots can play on campus, providing a low-barrier means for students to cope
with mental health challenges. \citet{yin2019deep} proposed Evebot,
an innovative sequence-to-sequence (Seq2Seq) based generative conversational
system designed for detecting negative emotions and preventing depression
through positively framed responses. The system integrates several deep learning
components, including a Bi-LSTM model for identifying negative user emotions, a
psychological counseling corpus, an anti-language Seq2Seq network, and a maximum
mutual information (MMI) model to enhance dialogue relevance.

Given that adolescents often avoid expressing negative emotions in face-to-face
settings, traditional methods of emotional support may be ineffective.
Therefore, Evebot leverages virtual platforms to identify early signs of
depression or anxiety, manage emotional states, and prevent the escalation of
mental health issues. In a one-month field study conducted on a campus platform,
the system demonstrated superior effectiveness in improving users' emotional
well-being compared to public chatbot baselines.

\subsection{Campus Service Chatbots}
Colleges invest a great deal of time and resources in optimizing their websites
to more effectively convey the key information about the institution and campus
resources. The institution's website functions as its "virtual image,"
projecting the specific face it wishes to present to the online community,
including prospective and current students, faculty and staff, parents, and
general users. Although these websites provide comprehensive information, they
lack the capability to offer personalized responses to users' inquiries. For
example, when a prospective student needs to learn the specifics of submitting
ACT scores, wishes to ascertain tuition and fee amounts, or is uncertain about
which parent's information to use to include in the FAFSA application, they must
navigate through multiple web pages to locate the answers. This process often
takes a considerable amount of time. However, sometimes due to unclear
information or the lack of personal interaction, users' questions remain
unanswered \citep{neupane2024questions}. Therefore, a chatbot can be a useful and
effective solution for students, providing them with immediate and up-to-date
information. "Lisa" is one of the examples of Campus Service chatbots. "Lisa"
is a chatbot designed to help students solve problems in their campus life by
providing information and services, answering questions 24 h a day, every day. 

Information provided by campus service chatbot is often generic, such as details
about university facilities, admissions processes, and course offerings.
However, chatbot assistants can be utilized in various contexts throughout the
academic year to address specific student needs. For instance, during the
application process, a chatbot can assist prospective students by guiding them
through enrollment procedures. The primary goal is to enable students to access
information quickly and efficiently, eliminating the need to search through
multiple web pages for answers to frequently asked questions. A chatbot serves
as a shortcut for obtaining information in a more accessible and natural manner.
It functions not only as an excellent guide for newcomers navigating the initial
steps into the university environment but also as a valuable resource throughout
students' entire campus experience \citep{dibitonto2018chatbot}. 

Many students do not attend classes every day. For them, it is very important to
be able to access updated information remotely. This information can be either
response-based (replies to specific requests) or push-based, such as relevant
updates and notifications to remind students of upcoming deadlines.
Response-based information typically provided by chatbots includes university
facilities, upcoming events, and academic information \citep{ranoliya2017chatbot}.
In essence, Campus Service Chatbots function as intuitive, always-available
digital assistants that enhance students' access to institutional support
throughout their academic journey.

\section{Technical Approach of Sentimental Analysis}
Sentiment analysis, or opinion mining, is an active area of study in the field
of natural language processing that analyzes people's opinions, sentiments,
evaluations, attitudes, and emotions via the computational treatment of
subjectivity in text. In this section, we will discuss the technical approaches
of sentimental analysis by looking into three common methods.

\subsection{Lexicon Methods}

Lexicon methods use a precompiled "sentiment lexicon" plus linguistic rules to
assign an overall sentiment score to a sentence or document. A sentiment lexicon
is a list of lexical features (e.g., words) which are generally labeled
according to their semantic orientation as either positive or negative. Although
one of the most reliable ways to create a sentiment lexicon is to create it
manually, it is also one of the most time consuming. Therefore, most of the
research on sentiment analysis rely heavily on pre-existing lexicons. Two main
kinds of lexicons exist currently: Polarity-based lexicons  and Valence-based
lexicons \citep{hutto2014vader}.

Polarity-based lexicons are lexicons in which words are categorized into binary
classes according to their context free semantic orientation. They are
straightforward to implement and enables low computational complexity. At the
meantime, they cannot distinguish between degrees of positivity- for example,
they would score "excellent" and "good" equally-and they require great effort to
maintain \citep{tripathi2016survey}. Some representatives of Polarity-based
lexicons are LIWC(Linguistic Inquiry and Word Count), GI (General Inquirer) and
Hu-Liu04. Valence-based lexicons are lexicons in which words are associated with
valence scores for sentiment intensity. By capturing fine-grained differences in
sentiment intensity by assigning each entry a continuous numerical score, they
allow models to distinguish between "excellent" and "good". Moreover, for
multi-meaning words, lexicons like SentiWordNet score each separately, ensuring
that a word's positive or negative intensity reflects its specific sense in
context. However, Valence-based lexicons also have the disadvantage of being
difficult to maintain \citep{yadollahi2018current} and ambiguated scoring in
continuous context. Some representatives of Valence-based lexicons are
ANEW(Affective Norms for English Words), SentiWordNet, SenticNet and NRC
VAD \citep{mohammad2018obtaining}. 

VADER (Valence Aware Dictionary for sEntiment Reasoning) is a sentiment analysis
tool that is able to capture both the polarity and intensity of sentiment.
Research has shown that the VADER lexicon performs exceptionally well in the
social media domain. Comparison of VADER with LIWC, GI, ANEW, SWN, SCN, WSD and
more tools shows that VADER performs better than those eleven highly-regarded
sentiment analysis tools \citep{hutto2014vader}.

\subsection{Machine-Learning Methods}

To address the time-consuming and hard-to-maintain issue of Lexicon methods, we
turn to Machine-Learning methods. Rather than relying on a manually created list
of sentiment-bearing words and complex linguistic rules, Machine-Learning
approaches build predictive models directly from labeled data. These methods
train classifiers--such as SVM or Naive Bayes--on labeled data using features like
n-grams and word embeddings, learning patterns that correspond to positive,
negative, and neutral sentiments. Once trained, the model assigns sentiment
scores to new texts without requiring manual lexicon updates. 

The benefits of applying machine learning are obvious. It is capable of learning
semantic features automatically and is better at capturing emotions in a long
context. However, it's training demands high computational resources (e.g.,
GPU/TPU) \citep{young2018recent} and it struggles to capture long-distance
dependencies and contextual meanings, and cannot automatically extract deep
semantic information \citep{ribeiro2016why}. 

\subsection{Deep-Learning Methods}

Different from Machine-Learning Methods which rely on manually crafted features
and shallow architectures, Deep-Learning Methods utilize multilayer neural
networks to automatically learn hierarchical representations from raw text data.
These neural networks are inspired by the structure of the biological brain.
Similar to the learning process of a biological brain, neural networks can learn
by adjusting the connection weights between neurons, which are information
processing units organized in units \citep{lecun2015deep}.

Based on network topologies, neural networks can generally be categorized into
feedforward neural networks, recurrent or recursive neural networks and the
combination of the two. In feedforward neural networks, data flows from the
input layer through one or more hidden layers to the output layer, with no
recurrent or feedback connections in between\citep{heaton-2017}. Typical types of
feedforward neural networks are CNN (Convolutional Neural Network), MLP
(Multilayer Perceptron) and Autoencoder. Recurrent neural networks (RNN) and
recursive neural networks introduce a recurrent connection in the hidden layer
between time step t and t-1, so that the network can retain the previous time
step's state, enabling it to model sequential data. 

Emerging as a powerful machine-learning technique and producing extraordinary
results in many application domains, deep learning has also shown great results
when applied to sentiment analysis, which includes learning automatically
without manual feature engineering and handling more complex sentence structures
such as negation and contrast better \citep{yang2016hierarchical}. On the other hand, due
to the black-box nature of deep learning models, interpretability is reduced,
making it difficult to identify which words or sentence segments the model
focuses on internally.

\section{Sentiment Analysis and Personalizational Strategies in Educational Chatbots}

\subsection{Overview}
In recent years, the application of sentiment analysis technology in educational
chatbots has gradually become a research hotspot, especially with the
development of multimodal sentiment recognition systems, which have
significantly improved AI's ability to understand learners' emotional states.
Traditional sentiment recognition primarily relies on textual information, which
has limitations in accurately identifying complex emotions, making it difficult
to achieve precise emotional adaptation in educational settings. 

According to \citet{heilala2024overview}, current educational
chatbots still primarily rely on text-generation models for sentiment analysis,
such as ChatGPT or similar large language models for interaction, while research
on integrating multimodal information such as speech, images, and physiological
signals remains limited. However, multimodal information (such as voice tone,
facial expressions, and behavioural responses) can enhance AI's ability to
understand, but its application in the educational field is still in its early
stages. The study emphasises that multimodal emotion recognition will be an
important direction for the future development of educational AI, but its
implementation still faces challenges such as model complexity and the
difficulty of data collection and integration. 

\citet{kovacs2024multimodal} further implemented a multimodal
emotion recognition framework based on text, audio, and visual data in an actual
system to enhance emotion recognition capabilities in human-computer
interaction. The results showed that the system significantly improved the
accuracy of chatbots in recognising user emotions and dynamically adjusted
response strategies based on emotional signals from different modalities. Such
systems are highly adaptable to educational scenarios, such as building virtual
teachers with emotional perception capabilities or question-answering robots,
thereby enhancing the personalisation and humanisation of learning interactions. 

In summary, research on emotion analysis in educational chatbots is evolving
from single-text modalities toward multi-modal fusion. Multi-modal emotion
recognition not only helps improve the understanding of deeper emotional states
but also provides technical support for the implementation of personalised
teaching strategies, signalling the development of educational AI toward smarter
and more human-centric directions.  

\subsection{Emotion Recognition Usages in Chatbots}

With the development of artificial intelligence technology, chatbots in the
education field have gradually gained the ability to dynamically adjust teaching
strategies based on learners' emotional states.

\citet{ma2021one} proposed a method for automatically constructing
implicit user profiles based on historical dialogue data. This method uses a
Transformer model to learn users' language styles and preferences from their
historical responses and construct dynamic user profiles. By introducing a
key-value memory network, the system can process current input. 

In addition to processing users' real-time emotions and providing flexible
responses, higher-level strategies include combining users' historical learning
records to generate more customised responses. After analyzing the learning
capabilities of users' multi-round dialogue data, the system can generate an AI
profile of the user, storing changes in the user's learning state as "long-term
memory" to predict their potential behaviour, preferences, and psychological
state, thereby providing more customised responses.

According to \citet{baradari2025neurochatneuroadaptiveaichatbot}, the NeuroChat system is a
neuro-adaptive AI tutor, which combines real-time electroencephalogram (EEG)
monitoring with generative AI technology. The system uses wearable EEG devices
to monitor learners' cognitive levels in real time and dynamically adjust the
complexity and pace of instructional content based on cognitive levels, thereby
achieving a personalised learning experience. Experimental results show that the
NeuroChat system has a significant effect on improving learners' cognitive and
subjective engagement levels. 

\subsection{The Role and Challenges of Emotion Recognition Chatbots in Educational Settings}

As artificial intelligence technology continues to mature, emotion recognition
chatbots are increasingly being applied in educational settings, demonstrating
significant value as teaching aids. According to \citet{arsad2024integrating},
artificial intelligence emotion recognition technology
holds promising application prospects in learning environments, enabling
real-time monitoring and assessment of learners' learning states, thereby
providing more personalised support and intervention for the learning process.
The integration of this technology helps enhance learners' motivation, emotional
regulation abilities, and ultimately their learning outcomes.

\subsubsection{Advantages of Chatbots in Educational Field}

Specifically, emotion recognition chatbots have three significant advantages.
First, \citet{siregar2024ai} point out that such systems can
serve as virtual tutoring assistants, providing learners with immediate feedback
and cognitive support through real-time dialogue to enhance learning initiative. 

Secondly, by continuously recording and analyzing learners' emotional states
during interactions through AI systems, teachers can adjust teaching methods and
pacing to achieve personalized instruction. 

Thirdly, AI chatbots can maintain learners' learning pace without teacher
intervention, thereby enhancing their self-directed learning abilities and
persistence. 

\subsubsection{Disadvantages of Chatbots in Educational Field}

However, this technology still faces significant challenges in educational
practice. According to \citet{arsad2024integrating}, the accuracy of emotion recognition
is one of the core technical challenges in current applications. Complex
emotional states are often difficult to accurately identify using single-modal
data. While multimodal fusion has improved accuracy, it remains constrained by
the generalization capabilities of algorithms. Additionally, the literature
highlights that the collection of personal data such as facial, voice, and
physiological information inevitably raises privacy protection and ethical
concerns, particularly in educational applications involving adolescents, where
caution is essential. 

\citet{siregar2024ai} further point out that AI chatbots lack the ability to
handle high-complexity teaching tasks and cannot fully replace human teachers'
instructional judgment and emotional care. Meanwhile, over-reliance on AI
technology may weaken social interaction between teachers and students,
affecting students' learning emotions and thereby having a negative impact on
their learning motivation. 

In summary, emotional recognition chatbots, as an important branch of
educational AI, have demonstrated positive effects in areas such as personalised
teaching, dynamic feedback, and learning state recognition. However, to achieve
widespread adoption in educational settings, continuous optimisation is needed
in dimensions such as recognition accuracy, data security, and ethical
regulation. 

\subsubsection{Feedback Adjustment Post-Emotion Recognition in Education}
With the development of artificial intelligence technology, chatbots in the
field of education have gradually acquired the ability to recognise learners'
emotional states and adjust feedback forms accordingly, thereby enhancing user
satisfaction and learning effectiveness. 

According to \citet{yin2024effects}, educational chatbots
can effectively reduce learners' negative emotions during interactions and
enhance their learning motivation by providing metacognitive feedback. The study
employed an experimental design to compare differences in emotional responses
and learning initiative between students receiving metacognitive feedback and
those receiving neutral feedback. The results showed that students in the
metacognitive feedback group exhibited higher levels of learning interest and
motivation. 

Additionally, \citet{han2022development} found in a study targeting nursing
students that educational programmes incorporating chatbots with personalised
feedback capabilities significantly improved students' learning interest and
self-directed learning abilities. Although no significant differences were
observed in knowledge mastery levels or clinical skills, students expressed high
satisfaction with the feedback provided by the chatbot, believing it enhanced
their learning experience. In summary, the application of emotion recognition
technology in educational chatbots not only enhances user satisfaction but also
boosts learning motivation and self-directed learning abilities, thereby
improving overall learning outcomes.

\section{Conclusion}

\subsection{Answers about research Questions}

\begin{itemize}
    \item \textbf{RQ1: Current State of Campus Chatbots}\\Campus chatbots are evolving
    into three key categories: academic assistance (e.g., EDUBOT for Q\&A and
    learning support), emotional support (e.g., Evebot for mental health
    monitoring), and campus services (e.g., Lisa for administrative queries).
    However, most remain rule-based or FAQ-driven, lacking adaptive emotional
    intelligence.

    \item \textbf{RQ2: Sentiment Analysis Techniques}\\There are some core
    techniques including Techniques span lexicon-based (e.g., VADER for social
    media), machine learning (e.g., SVM for labeled data), and deep learning
    (e.g., Bi-LSTM/Transformer for context-aware modeling). Multimodal fusion
    (text + audio + visual) now enhances accuracy but still faces algorithmic
    generalization limits.

    \item \textbf{RQ3: Impact on User Experience \& Learning}\\Emotion-aware
    chatbots significantly improve satisfaction and outcomes. Examples include:
    \begin{itemize}
        \item Metacognitive feedback reducing negative emotions.
        \item EEG-driven tutors (e.g., NeuroChat) adapting content to cognitive
        states.
    \end{itemize}
\end{itemize}

\subsection{Recommendations}

\begin{enumerate}
    \item \textbf{For researchers}:\\Prioritize the construction of culturally diverse
    datasets to reduce bias in emotion recognition.
    \item \textbf{For educators}:\\Adopt hybrid deployment: chatbots for routine
    queries like admissions and human instructors for complex mentoring.
    \item \textbf{For developers}:\\Design modular architectures like
    Rasa/Dialogflow, enabling seamless integration of sentiment analysis modules
    with campus systems.
\end{enumerate}

Emotion-aware campus chatbots are becoming a key tool in promoting personalized
education. Not only can they answer academic questions and provide guidance on
campus life details like an 'all-weather learning partner,' but they can also
act as guardians, sensing students' level of knowledge acquisition and emotional
fluctuations caused by academic pressure, and providing timely comfort,
encouragement, or resource guidance. This ability to deeply understand and
proactively care for students allows technology to truly serve their growth,
rather than merely conveying information.

\bibliography{references}

\end{document}